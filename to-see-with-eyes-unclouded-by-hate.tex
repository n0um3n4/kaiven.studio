\documentclass[10pt,twocolumn]{article}

\usepackage[margin=1.8cm]{geometry}
\usepackage{setspace}
\usepackage{parskip}
\usepackage[T1]{fontenc}
\usepackage{lmodern}

\setlength{\columnsep}{0.8cm}
\setstretch{1.1}

\begin{document}

\begin{center}
    {\LARGE\bfseries To See With Eyes Unclouded By Hate}
\end{center}

\vspace{0.5cm}

I have many reasons to hate myself.  
Many reasons to hate my own species.

I have seen what humans are capable of when no one is watching.  
I have seen cruelty practiced not out of hunger, not out of necessity, but out of boredom, convenience, and profit.  
I have seen pain turned into a tool. Suffering turned into a system.

I have watched the planet bleed quietly while we argued about comfort.  
Hundreds of species erased.  
Not sacrificed.  
Erased.

We like to tell ourselves it was survival.  
It wasn't.

It was appetite.

So I grieved.

Not dramatically.  
Not loudly.

I grieved the way one grieves when there is no funeral, no closure, no single body to bury.  
I grieved by changing myself.  
By refusing meat.  
By trying to live lighter.  
By trying, foolishly, desperately, to become harmless.

At one point, I even made my dogs vegetarians.  
I told myself it was compassion.  
It was grief pretending to be ethics.  
I reversed it later, when I realized that purity, taken too far, becomes another kind of violence.

That's what this planet does to you.  
It twists even your attempts at goodness into something brittle.

I have walked the long arc between science and mysticism.  
I have studied neurons and rituals.  
Equations and prayers.  
Evolution and myth.

And somewhere along that dance, something became clear to me:

There are other forces at work here.

Not metaphorical ones.  
Not psychological projections.

Forces.

I have seen goodness too.  
Real goodness.  
Quiet goodness.  
People who choose kindness when it costs them something.  
People who break cycles they did not create.  
People who carry pain without passing it forward.

That's what saved me from hate.

Because hate is easy.  
Hate is lazy.  
Hate is the fastest way to stop seeing.

One day, I will leave this plane of existence.  
Maybe I already have.  
Maybe this is just the echo of a place I no longer fully inhabit.

What I know - what I feel with a certainty deeper than belief - is this:

There is a war.  
Not one with banners or bullets.  
A war between ways of being.  
Between dimensions of perception.  
Between unconscious repetition and conscious choice.

And at the front of it - 

It's me.

Not as a hero.  
Not as a savior.

As a witness.

The question that haunts me isn't \textit{whether the war is real}.  
It's simpler than that.

Am I alone?

Or am I finally seeing - 

Not through rage.  
Not through despair.  
Not through the inherited bitterness of a wounded species.

But with eyes unclouded by hate.

And if I am...

Then maybe that, too, is a kind of duty.

\end{document}
