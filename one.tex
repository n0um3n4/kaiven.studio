\documentclass[12pt,oneside]{book}

\usepackage[utf8]{inputenc}
\usepackage[T1]{fontenc}
\usepackage{lmodern}
\usepackage{titlesec}
\usepackage{geometry}

\setlength{\parindent}{0pt}
\titleformat{\chapter}[display]
  {\normalfont\bfseries\Large\centering}
  {\chaptertitlename\ \thechapter}{20pt}{\Large}

\newcommand{\dialogue}[3]{
	\begin{quote}
		\centering
		\vspace{0.25cm}
		\textit{#2} \\ \textbf{\footnotesize{#1}} \\\textit{\footnotesize{#3}}
		\vspace{0.25cm}		
	\end{quote}
}

\begin{document}

\pagestyle{empty}

\begin{titlepage}
	\centering
    \vspace*{\fill}
    {\Huge\bfseries One \\}
    \vspace*{\fill}
\end{titlepage}

Like most souls dragged into this mess of a world, my childhood carried legacies I never consented to.

\vspace{0.618cm}

My mother was seventeen — barely more than a child herself — when she carried me.  
A girl unloved. Untouched by kindness. Spoken to in commands rather than compassion.  
Her own mother had cast her aside.  
My father? A phantom. A rejection so complete it lingered like smoke in her lungs.

She once told me something I still can’t let go of.  
That while I grew inside her, I flinched whenever she cried.  
Moved when she felt alone.  
As if, even before birth, I already knew the taste of abandonment.

\vspace{0.618cm}

I was six the first time I met death.  
Not heard about it — met it.  
I didn’t understand it then. I only remember the weeping, the silence afterward, the hole no one knew how to name.

My grandmother — my tether to anything warm — died of cancer.  
I watched it take her piece by piece.  
I watched her disappear while she was still breathing.

That broke something in me.

\vspace{0.618cm}

Crying became as natural as blinking.  
Depression nested. Took root.  
The suicidal thoughts came later — not as drama, but as logic.  
A conclusion drawn from a life that made no promises  
and still found a way to break them.

\vspace{0.618cm}

I’m older now.  
I’ve read the books.  
Chewed the ideas until they lost their flavor.  
Philosophy? It’s like drinking sand when you’re thirsty.  
It looks like it might help — until it doesn’t.

\vspace{0.618cm}

And still the question echoes:

What’s the point?

\vspace{0.618cm}

Right now, I’m standing at the edge of a cliff.  
Literal? Maybe.  
Metaphorical? Certainly.

I feel the wind — cold, clean.  
One step, and it’s finished.  
No more hunting for answers in pages.  
No more trying to pour meaning into a cup cracked all the way through.

\vspace{0.618cm}

You’re probably thinking it.  
\textbf{Love}.  
Or \textbf{God}.  
Or some shimmering thing we pretend will save us.  
\textit{Hope}.  
That lie we tell ourselves to keep the blade from pressing too deep,  
or the pills from going down too fast.

I used to believe it.

Or at least I tried.

Maybe you still do.  
And if you do... I’m glad.  
Illusions can be beautiful.  
Sometimes they’re the only things that get us through.

But for me?

There’s nothing left.

Just a count.

The count.

\vspace{0.618cm}

Three...

\vspace{0.618cm}

Two...

\vspace{0.618cm}

...

\vspace{0.618cm}

Wait.

\vspace{0.618cm}

That’s the thing about cliffs.

They demand a decision.

And sometimes, in that final second before the step,  
you remember the weight of existence.

Not as a burden.

But as proof.

Proof that you were here.

That maybe — 

just maybe — 

there is still something worth finding  
on the other side of the ledge.

\vspace{0.618cm}

One...
\end{document}
