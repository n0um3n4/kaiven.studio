\documentclass[10pt,twocolumn]{article}

\usepackage[utf8]{inputenc}
\usepackage[T1]{fontenc}
\usepackage[margin=1.6cm]{geometry}
\usepackage{microtype}

\setlength{\columnsep}{0.7cm}
\setlength{\parindent}{0pt}
\setlength{\parskip}{0.6em}

\title{\vspace{-1.2em}\Large The World Of Men\par}
\author{}
\date{}

\begin{document}
\maketitle
\vspace{-1.0em}

The forest was old, and its leaves spoke softly one to another in a tongue that men scarcely hear. There, beneath a roof of green and a net of pale light, an Elf walked without haste, and beside him a Man went as one who has come far and yet is not certain why.

\textbf{Man:} The world of Men is loud. Even when it is quiet, it is loud within. I sought peace, and found only more voices. Tell me, if you will: where has peace gone?

\textbf{Elf:} Peace has not gone. It is simply no longer sought with patience. Men hunt it as they hunt a fleeing thing, and so they drive it farther away. Yet it is not peace alone that troubles you. You carry another question, unspoken.

\textbf{Man:} I have heard it said, even among Men: that there are no men anymore. Only grown bodies with boyish hearts. I did not believe it. And yet, I look upon my people, and I am afraid.

\textbf{Elf:} Fear is not always a liar. Often it is the first messenger of truth. Sit. Listen to the leaves a moment, and the answer will come more gently.

They sat upon a fallen trunk. The Elf laid his hand upon the bark as one greeting an old friend.

\textbf{Elf:} In your tales, what is a man?

\textbf{Man:} One who is strong. One who provides. One who endures.

\textbf{Elf:} Those are shadows of the thing, not the thing itself. Strength is not the root, it is the fruit. And endurance without wisdom is merely a stone that does not move, even when it should.

\textbf{Man:} Then what is the root?

\textbf{Elf:} Governance. The mastery of the inward realm. A man is one who has learned to rule his own heart: not by denying it, nor by being dragged behind it like a captive, but by knowing it and guiding it. Pain, anger, desire, grief, these are fires. In a hearth they warm, in dry grass they devour.

\textbf{Man:} You speak as though Men have forgotten the hearth.

\textbf{Elf:} Many have. They have forgotten the old passages, and so they remain unpassed.

\textbf{Man:} The rites? The proving? Those were barbarous, some say. We have grown beyond them.

\textbf{Elf:} Have you? When a bridge is broken, the river does not cease to flow. Men still must cross. If you remove the crossing, they will attempt the waters unprepared, or pretend the far bank is not there. The need did not vanish, only the form did.

\textbf{Man:} And so we have boy-men.

\textbf{Elf:} And so you have men-children: quick to wrath, hungry for dominion, and fearful beneath their shouting. They have not met their own sorrow and made peace with it. Therefore they seek to place their unrest into the world, as one flings a burning brand from his own hand.

\textbf{Man:} Is that why there is violence? Why the cruelties I cannot speak of without shame?

\textbf{Elf:} It is a root among roots. A heart unruled will seek a smaller heart to rule. Not always, but often. Where there is no inward honor, outward power becomes a toy. And toys, in the hands of children, are broken.

The Man looked down, and his fingers dug into the moss.

\textbf{Man:} We say wars are caused by men.

\textbf{Elf:} Wars are kindled by those who cannot endure themselves. Not by the strong, but by the unripe. A true man seeks peace, not as softness, but as steadiness. He has faced his own darkness, and therefore does not need to cast it upon others.

\textbf{Man:} Yet the world teaches the opposite. It teaches appetite. It teaches triumph. It teaches that to feel is weakness.

\textbf{Elf:} And so feelings become a secret tyrant. What is denied does not die, it returns masked. Anger that is not understood becomes cruelty. Fear that is not faced becomes control. Grief that is not wept becomes hardness. Thus do men-children become dangerous, for they are governed by shadows.

\textbf{Man:} You spoke of a teacher. The sternest teacher. Death.

\textbf{Elf:} Aye. Death is the ultimate tutor of life. In our long years we look upon it often, not with haste, but with clear eyes. For it says, "You are not endless. Therefore choose well". But many among Men have hidden death behind walls of noise and distraction. They do not sit with it,  they do not let it speak. And so they live as though there were always another day to mend what they break.

\textbf{Man:} Then the world of Men is in danger because we have forgotten to end?

\textbf{Elf:} Because you have forgotten to \emph{remember} the end. The remembrance is not despair, it is the beginning of reverence. When a man knows his days are numbered, he learns what is worthy of his strength. He becomes slower to harm and quicker to guard. He becomes careful with love, as one is careful with a flame in high wind.

\textbf{Man:} And women? You spoke once that women know pain early.

\textbf{Elf:} Many do. Not all in the same manner, yet many. They are acquainted with the deep waters sooner than men. Therefore they often live nearer to their emotions, not as scholars peering through glass, but as swimmers within the stream. They do not always need to \emph{explain} reality, because they have been compelled to \emph{experience} it with their whole being.

\textbf{Man:} Then why do men so often misunderstand them?

\textbf{Elf:} Because men, when unmade, mistake feeling for confusion, and depth for threat. But a grown man does not fear the sea, he learns its tides. He does not demand that the river become stone, he builds a bridge, and honors the water.

The Man was silent for a long while. Above them a bird called, and the forest answered.

\textbf{Man:} You said earlier that a man seeks peace. Yet I have also heard men speak of \emph{owning} a woman. The word turns my stomach, and still I know what they mean. They mean to protect. They mean to cherish. But it becomes something darker.

\textbf{Elf:} Words are blades: they cut according to the hand that holds them. If a man says, "She is mine", and means, "I will bind her, I will diminish her, I will feed my hunger upon her freedom", then he is not a man but a thief. Yet if he means, "I am bound to her welfare, I will nurture, I will guard, I will be loyal, I will not abandon", then he speaks of a vow, not a possession.

\textbf{Man:} How shall a man protect without imprisoning?

\textbf{Elf:} By learning that protection is service, not dominion. By strength that does not demand payment. By a loyalty that does not bargain. By hands that are gentle because they \emph{could} be harsh, and choose not to be. No true man will touch a woman in violence, for his power is not for taking, but for keeping safe what is dear.

\textbf{Man:} And what of honor? It is spoken of, and yet it seems thin as a painted shield.

\textbf{Elf:} Honor is not an ornament. It is the daily craft of self-rule. Truth spoken when it costs. Restraint held when wrath is easy. Work done when no one sings of it. The man-children you fear are weak of mind, and so they become weak of body also, for the spirit that cannot command itself cannot long command the hands. They seek shortcuts to strength, and find only performance.

\textbf{Man:} Then if we desire a better world, what must be done?

\textbf{Elf:} Men must become men. Not in boast, nor in cruelty, nor in the empty theater of hardness, but in the long, humble labor of becoming: to face fear without fleeing into noise, to bear pain without turning it into a weapon, to learn tenderness without surrendering strength, to remember death without falling into despair, and to choose peace, not once, but daily.

\textbf{Man:} That sounds like a road with no end.

\textbf{Elf:} All worthy roads are so. Yet they lead somewhere. The danger of your world is not that Men have become weak, but that they have forgotten what strength is for. Strength is not for domination. It is for protection, for service, for keeping faith. When Men forget this, the world sickens, for the young are unguarded, the women are endangered, and the land itself is treated as a thing to be consumed.

\textbf{Man:} And if we remember?

\textbf{Elf:} Then peace returns, as dawn returns: not because you seize it, but because you make a place for it. You clear the brambles. You tend the fire. You keep the watch.

The Man rose, and the forest seemed to grow wider around him, as if a door had opened in his mind.

\textbf{Man:} Teach me the first step.

The Elf looked up through the leaves where the light fell like scattered silver.

\textbf{Elf:} Be silent long enough to hear yourself. Then be brave enough to meet what you hear. And do not turn away from the thought of your ending,  let it make your love more faithful.

They walked on beneath the boughs, and the forest whispered of old truths, patient and unashamed.

\end{document}
