\documentclass[12pt,oneside]{book}

\usepackage[utf8]{inputenc}
\usepackage[T1]{fontenc}
\usepackage{lmodern}
\usepackage{titlesec}
\usepackage{geometry}

\setlength{\parindent}{0pt}
\titleformat{\chapter}[display]
  {\normalfont\bfseries\Large\centering}
  {\chaptertitlename\ \thechapter}{20pt}{\Large}

\newcommand{\dialogue}[3]{
	\begin{quote}
		\centering
		\vspace{0.25cm}
		\textit{#2} \\ \textbf{\footnotesize{#1}} \\\textit{\footnotesize{#3}}
		\vspace{0.25cm}		
	\end{quote}
}

\begin{document}

\pagestyle{empty}

\begin{titlepage}
	\centering
    \vspace*{\fill}
    {\Huge\bfseries The Shaman \\}
    \vspace*{\fill}
\end{titlepage}

For as long as I can remember, stories have lived inside me.

For a long time, I thought it was the videogames.  
The board games.  
The little plastic warriors lined up on the floor, waiting for battles I invented for them.  
That explanation made sense.  
I spent hours inside imagined worlds, letting them close around me like armor.

But not long ago, I remembered something older.  
Quieter.  
Something buried deeper than nostalgia.  
Something I hadn’t touched in years.

\vspace{0.618cm}

My childhood wasn’t easy.  
Not tragic in the way people like to dramatize,  
but heavy — the kind of weight that settles into your bones and never fully leaves.

I was five autumns old when I first understood death.  
Not the concept.  
The fact.

That simple, cold realization:  
things die.  
People disappear.  
And some questions are never answered.

\vspace{0.618cm}

I grew up on the hills of the first city founded after the Spanish arrived in the Americas.  
A place where history is not taught — it seeps.  
Carved into stone walls.  
Pressed into the earth.  
A place where, if you stay quiet long enough, the past begins to whisper.

\vspace{0.618cm}

And every week, a man would walk into our neighborhood.

Not a local.  
Not someone anyone really knew.  
Just… a man.

Older than old.  
Thin as driftwood.  
Eyes sharp in a way that made you feel he was always a few steps ahead —  
inside a story you didn’t yet know you were part of.

\vspace{0.618cm}

He carried a bag of glass spheres\footnote{\textit{The View-Master is a stereoscope device that uses reels of 3D images, originally designed to provide a virtual travel experience and later popular as a toy for children. It was first introduced in 1939, taking advantage of the newly available Kodachrome color film to display high-quality, small color images in 3D}}.

Perfect little orbs.  
Not cheap crystal.  
Something denser.  
Heavier.

As if they were filled with time.

For a few silver coins, he would let you choose one.  
You’d sit there, fingers still warm from holding it,  
and he would begin to speak.

Not loudly.  
Not theatrically.

Just clearly.  
With rhythm.

As if the story had been waiting,  
and he was merely the one chosen to let it pass through him.

And the orb would glow.

Not like a lamp.  
Like memory.

Like the flicker you see behind your eyelids  
when you try to remember someone’s face  
after they’re gone.

\vspace{0.618cm}

Visions.  
Forgotten kingdoms.  
Floating mountains.  
Deserts made of singing glass.

And always —  
the quiet certainty that what he was telling you was not fiction.

It was history.  
Someone’s history.

\vspace{0.618cm}

I never learned his name.  
Never knew where he came from  
or where he went when he disappeared down the dusty road.

I just called him \textit{\textbf{The Shaman}}.

It felt right.

He didn’t sell stories.  
He summoned them.

\vspace{0.618cm}

And maybe that’s why I tell stories now.

Not for glory.  
Not for recognition.

But because once —  
when I was just a lonely child sitting on a crumbling hill —  
someone showed me that stories can carry you across time.

Not to erase pain.  
But to understand it.  
To give it shape.  
To make it bearable.

\vspace{0.618cm}

He gave me a sphere once.  
Said I didn’t have to pay.  
Told me to keep it  
until I had stories of my own to share.

I still have it.

It doesn’t glow anymore.

But sometimes, when I’m writing,  
I catch it shifting in the light —  
just a little.

As if it’s listening.

Waiting  
for the next tale…
\end{document}
