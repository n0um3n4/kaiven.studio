\documentclass[a4paper,12pt]{book}
\usepackage[spanish]{babel}
\usepackage{fontspec}   
\usepackage{geometry}   % Page layout
\usepackage{enumitem}   % Custom lists for script format
\usepackage{xcolor}       % for color definitions
\usepackage{pagecolor} 
\geometry{top=2cm, bottom=2cm, left=2.5cm, right=2.5cm}

% Define script formatting
\newcommand{\scene}[1]{\section*{#1}}  % Scene headers
\newcommand{\character}[1]{\textbf{#1}:}  % Character names in dialogue
\newcommand{\action}[1]{\textit{(#1)}}  % Scene descriptions or actions

\begin{document}
\pagecolor{violet}
\begin{titlepage}

\centering
\vspace*{8cm}
{\Huge\bfseries \textcolor{white}{TETEO} \\[4.5cm]}
\vfill
\end{titlepage}

\newpage

\nopagecolor % optionally resets page color after this page

\chapter*{Corazón de Rey}
\addcontentsline{toc}{chapter}{Corazón de Rey}

\scene{Premisa}

\action{Este es el viaje de Teteo.
No un viaje para aprender cosas,
sino para recordar algo que ya está dentro de él.
\\
Teteo caminará acompañado por su 'amá.
No para que ella lo guíe con respuestas,
sino para que lo sostenga mientras aparecen las preguntas.
\\
Teteo aprenderá qué significa ser un dios.
No será sencillo.
No será suave.
Habrá dolor.
Habrá confusión.
Habrá momentos en los que querrá destruir lo que ama.
\\
Al final, Teteo comprenderá que un dios con emociones
no es un error,
sino una responsabilidad.
\\
Aprenderá que el poder sin cuidado es ruido,
y que la bondad no nace del miedo,
sino del control.
\\
Su 'amá lo ama profundamente.
Es frágil como un colibrí,
y aun así,
es lo suficientemente fuerte
como para preparar a su hijo para perderla.
\\
Su destino es triste.
Y necesario.
\\
Porque Teteo no puede convertirse en el dios que debe ser
sin antes aprender a cuidar.}

\scene{Introducción}

\character{Teteo} 'Amá… ¿somos dioses?

\action{'Amá sonríe.
No responde de inmediato.
Como si la pregunta necesitara respirar primero.}

\character{'Amá} Vamos a averiguarlo.

\scene{El Jardín}

\action{Teteo se encuentra en un lugar lleno de vegetación.
El mundo parece escuchar.
\\
Observa un nido de hormigas.
Durante un largo rato, no hace nada.
Solo mira.
\\
Después, juega con ellas.
Las sigue.
Las toca.
\\
Más tarde, sin entender del todo,
les arranca las antenas.
Las enfrenta.
Rompe nidos.
Las obliga a trabajar para él.
\\
No por maldad.
Por curiosidad.
\\
Su 'amá observa todo en silencio.}

\vspace{0.618cm}

\character{'Amá} Teteo… ven.
\action{con una voz suave,
pero atravesada por algo que duele}

\vspace{0.618cm}

\action{Teteo se acerca,
con tierra en las manos}

\vspace{0.618cm}

\character{'Amá} Un día ya no estaré contigo.
Y antes de irme,
quiero ayudarte a ser un buen dios.

\vspace{0.618cm}

\character{Teteo} ¿Entonces sí somos dioses?
\action{pregunta con los ojos abiertos,
como si eso lo cambiara todo}

\vspace{0.618cm}

\character{'Amá} Sí, Teteo.
Pero ser un dios no es fácil.

\vspace{0.618cm}

\action{Teteo guarda silencio.
Escucha.}

\vspace{0.618cm}

\character{'Amá} Un dios debe aprender a servir, mi'jo.
Mira las hormigas.
\action{señala a un grupo que sigue trabajando,
sin haber sido tocado}

\vspace{0.618cm}

\character{'Amá} Mira cómo cada una hace algo distinto.
Algunas van solas.
Otras juntas.
Pero todas cuidan algo más grande que ellas mismas.

\vspace{0.618cm}

\action{Teteo observa,
como si las viera por primera vez}

\vspace{0.618cm}

\character{'Amá} Un dios, Teteo,
tiene corazón de Rey.

\vspace{0.618cm}

\character{Teteo} ¿Corazón de Rey?

\vspace{0.618cm}

\character{'Amá} Sí.
No son los poderes lo que hace grande a un dios.
Un dios es grande porque cuida.
No manda.
Escucha.
No destruye.
Protege.
No brilla para que lo miren.
Brilla para que otros no se pierdan.

\vspace{0.618cm}

\character{Teteo} ¿Un dios… es como un jardín?

\vspace{0.618cm}

\character{'Amá} Así es.
\action{dice sonriendo}

\vspace{0.618cm}

\character{'Amá} Un dios es como un jardín para las hormigas.
No para ser servido,
sino para servir con amor.

\vspace{0.618cm}

\character{Teteo} ¿Y si un día me enojo
y quiero destruir el jardín?

\vspace{0.618cm}

\character{'Amá} Entonces respiras.
Escuchas tu corazón.
Y te preguntas:
¿estoy cuidando
o estoy mandando?

Porque un dios con corazón de Rey
elige cuidar,
incluso cuando está enojado.
\action{'Amá abraza a Teteo}

\vspace{0.618cm}

\action{Teteo corre hacia las hormigas.
Ayuda a reconstruir los nidos.
Con cuidado, devuelve las antenas.
Usa sus poderes para hacer caminos,
puentes pequeños,
senderos seguros
a lo largo del jardín.}

\vspace{0.618cm}

\character{Teteo} ¿A dónde irás, 'Amá?
\action{pregunta sin dejar de construir}

\vspace{0.618cm}

\character{'Amá} Todo a su tiempo, Teteo.
Todo a su tiempo.


\chapter*{Guerrero en Equilibrio}
\addcontentsline{toc}{chapter}{Guerrero en Equilibrio}

\scene{El Jardín}

\action{Teteo vuelve al Jardín.
Cada día observa a las hormigas trabajar juntas.
Construyen túneles.
Buscan alimento.
Se cuidan unas a otras.
\\
Cuando aparecen enemigos,
Teteo levanta las manos.
Rayos brotan de sus dedos.
Protege a sus hormigas con fuego y luz.
\\
A veces incluso toma prestado al sol
para quemar aquello que se acerca demasiado.}

\vspace{0.618cm}

\character{Teteo} ¡Nadie tocará a mis hormigas!
¡Yo soy su protector!
\action{dice con el pecho inflado,
orgulloso}

\vspace{0.618cm}

\action{El tiempo pasa.
Los enemigos desaparecen.
Uno a uno.
\\
Pero el Jardín comienza a cambiar.
La tierra se agrieta.
Las plantas se marchitan.
Los caminos se rompen.
\\
Algo está fuera de equilibrio.}

\vspace{0.618cm}

\character{Teteo} ¿Qué pasa?
\action{pregunta,
con la voz más pequeña,
mientras 'Amá se acerca}

\vspace{0.618cm}

\character{'Amá} Estás luchando como un guerrero, Teteo.
Pero has olvidado algo importante.

\vspace{0.618cm}

\character{Teteo} ¡Estoy protegiendo!
¡No dejo que nada las lastime!
\action{exclama,
con tristeza y desesperación}

\vspace{0.618cm}

\character{'Amá} Lo sé, mi'jo.
No dudo de tu amor.
Pero si deseas ser un verdadero guerrero,
hay una lección que debes aprender.

\vspace{0.618cm}

\character{Teteo} ¿Qué, 'Amá?
¿Qué estoy haciendo mal?
\action{pregunta sollozando}

\vspace{0.618cm}

\character{'Amá} Sin quererlo,
también estás evitando que aprendan.
Que se adapten.
Que enfrenten lo que la vida les trae.
\action{hace una pausa
y mira el Jardín}

\vspace{0.618cm}

\character{'Amá} El verdadero guerrero no es un muro
que detiene todo.
Es un guardián del equilibrio.

\vspace{0.618cm}

\action{Teteo frunce el ceño,
confundido}

\vspace{0.618cm}

\character{Teteo} ¿Quieres decir…
que debo dejarlas enfrentar peligros?

\vspace{0.618cm}

\character{'Amá} No todos los peligros son enemigos.
Algunos enseñan.
Algunos fortalecen.
Y otros simplemente forman parte
del orden del mundo.

\vspace{0.618cm}

\action{Teteo observa con atención.
Las arañas que había eliminado
comían insectos que enfermaban a las plantas.
Los pájaros dejaban caer semillas nuevas
sobre la tierra.
\\
El Jardín intentaba sanar.
\\
Teteo siente un nudo en el pecho. Había creído que proteger significaba alejar todo lo que doliera.}

\vspace{0.618cm}

\action{'Amá toma la mano de Teteo}

\vspace{0.618cm}

\character{'Amá} El guerrero no lucha contra la vida.
Aprende a discernir
cuándo actuar…
y cuándo confiar.

A veces,
proteger
es soltar con fe.

\action{Teteo suspira.
Y comprende,
sin poder explicarlo,
que no solo debe existir equilibrio afuera,
sino también dentro de él.}


\chapter*{Ojos de Mago}
\addcontentsline{toc}{chapter}{Ojos de Mago}

\scene{El Jardín}

\character{Teteo} Todo parece igual…
Todos los días.
Las hormigas trabajan,
comen,
duermen.
¿Eso es todo?
\action{dice suspirando,
como si algo le faltara}

\vspace{0.618cm}

\character{'Amá} ¿Quieres ver la magia, Teteo?

\vspace{0.618cm}

\action{Los ojos de Teteo se abren,
como si alguien hubiera dicho una palabra prohibida}

\vspace{0.618cm}

\character{Teteo} ¿¡Magia!?
¡Sí!
¿Puedo lanzar rayos de colores?
¿Puedo hacer flotar montañas?

\vspace{0.618cm}

\action{'Amá sonríe,
pero no responde de inmediato}

\vspace{0.618cm}

\character{'Amá} La magia verdadera no siempre se muestra.
A veces…
se siente.
Otras veces,
se esconde
hasta que alguien aprende a mirarla.

\vspace{0.618cm}

\action{'Amá vuelve a sonreír}

\vspace{0.618cm}

\action{'Amá camina por el Jardín.
Se detiene.
Recoge una piedra pequeña.
Se la pone en la mano a Teteo.}

\vspace{0.618cm}

\character{Teteo} Esto…
¿es magia?
\action{pregunta decepcionado,
mirando la piedra gris}

\vspace{0.618cm}

\character{'Amá} Sí.
Pero solo si aprendes a ver
con ojos de Mago.

\vspace{0.618cm}

\action{Teteo frunce el ceño.
Confundido,
camina hasta un nido de hormigas.
Coloca la piedra junto al camino.
Luego se esconde para observar.}

\vspace{0.618cm}

\action{Una hormiga se detiene frente a la piedra.
La rodea.
Otra sube encima.
Desde ahí,
avisa a las demás.
\\
En poco tiempo,
las hormigas cambian su ruta.
Usan la piedra como puente.}

\vspace{0.618cm}

\character{Teteo} ¿Ellas vieron algo
que yo no?

\vspace{0.618cm}

\character{'Amá} La magia no está en la piedra, Teteo.
Está en cómo la miras.

\vspace{0.618cm}

\action{Teteo levanta la piedra.
Ya no es solo gris.
Ve colores escondidos.
Grietas como mapas.
Pequeños cristales
que brillan con la luz.
\\
Algo ha cambiado.
No en la piedra…
\\
En él.}

\vspace{0.618cm}

\character{'Amá} El Mago transforma el mundo
no con hechizos…
sino con sabiduría.

Donde otros ven rutina,
el Mago ve misterio.
Donde otros ven caos,
el Mago ve patrones.
Y donde otros ven una simple piedra…
el Mago ve una puerta.

\vspace{0.618cm}

\action{Teteo cierra los ojos.
Respira.
\\
Cuando los abre,
ve a las hormigas como un lenguaje.
A las sombras como una danza.
Al viento como un susurro
de algo más grande.
\\
Y por primera vez,
siente que el universo entero
le está hablando.
\\
No necesita rayos.
No necesita fuego.
\\
Ahora,
tiene los Ojos del Mago.}


\chapter*{Amante No Pose}
\addcontentsline{toc}{chapter}{Amante No Pose}

\scene{El Jardín}

\action{Un día, Teteo observa el Jardín.
Lo conoce todo:
los túneles de las hormigas,
el reflejo del rocío,
el canto de las hojas al caer la tarde.
\\
Y aun así,
ese día ve algo que nunca había visto.
\\
No es una forma.
No es una cosa.
\\
Es… un color.}

\vspace{0.618cm}

\character{Teteo} ¿Qué es esto?
\action{susurra,
como si temiera que el color pudiera oírlo}

\vspace{0.618cm}

\action{Sigue el rastro del color.
Flota como una pluma en el viento.
Lo guía hasta un rincón del cielo
que no siempre está ahí,
donde las nubes son tan suaves
que parecen guardar secretos.
\\
Ahí conoce a Liria,
una pequeña diosa.}

\vspace{0.618cm}

\action{Liria no habla mucho.
Solo sonríe
y teje pétalos con la luz.
\\
Pero su presencia…
hace que el mundo
brille distinto.}

\vspace{0.618cm}

\action{Teteo no entiende lo que siente.
A veces quiere estar cerca todo el tiempo.
Otras, quiere correr y gritar sin saber por qué.
A veces se enoja sin motivo.
Otras, se queda quieto,
solo para verla respirar.}

\vspace{0.618cm}

\action{Una tarde,
Teteo le trae un regalo:
un pajarito.}

\vspace{0.618cm}

\character{Teteo} Mira, ten esto.

\vspace{0.618cm}

\action{Liria sonríe.
Abre las manos.
Y lo deja volar libre.}

\vspace{0.618cm}

\character{Teteo} ¿No lo quieres?
\action{pregunta dolido,
confundido}

\vspace{0.618cm}

\character{Liria} El pajarito no es mío.

\vspace{0.618cm}

\action{Teteo guarda silencio.
Algo le duele en el pecho.
\\
Y sin embargo…
\\
algo también brilla.}

\vspace{0.618cm}

\action{'Amá ve regresar a Teteo.
Nota que su hijo camina distinto.
Como si algo dentro de él
se hubiera movido de lugar.}

\vspace{0.618cm}

\character{'Amá} Has sentido el fuego,
hijo mío.

\vspace{0.618cm}

\character{'Amá} Veo que estás herido.
Veo que deseas poseerlo todo.

\vspace{0.618cm}

\character{Teteo} ¿Por qué no puedo tenerlo todo?

\vspace{0.618cm}

\character{'Amá} El fuego que sientes
no quiere poseer.
Quiere sentirlo todo.

No encierra.
Acompaña.
No exige.
Ofrece.

Y cuando el fuego es verdadero…
deja libre
incluso aquello que más desea.

\vspace{0.618cm}

\action{Teteo mira al cielo.
El pajarito vuela,
ya no es de nadie,
y ahora es parte del universo.
\\
Dentro de él,
algo se vuelve más suave…
\\
y al mismo tiempo,
más fuerte.}

\vspace{0.618cm}

\action{Teteo vuelve a ver a Liria.
Esta vez no lleva regalos.
Solo su presencia.
\\
Juntos crean un color nuevo.
No tiene nombre,
pero vibra en sus corazones.}

\vspace{0.618cm}

\action{Entonces Teteo comprende,
sin poder explicarlo,
que el amor…
\\
es aquello que nos atraviesa
y nos transforma
sin pedir permiso.}

\chapter*{Noche Sin Fin}
\addcontentsline{toc}{chapter}{Noche Sin Fin}

\scene{El Jardín}

\action{'Amá está sentada en el Jardín.
Espera a Teteo.
\\
En su rostro vive algo extraño:
tristeza y alegría
ocupando el mismo espacio.}

\vspace{0.618cm}

\action{Teteo llega cantando.
Silba una melodía sin nombre.
La ve sentada
y corre hacia ella,
lleno de luz.}

\vspace{0.618cm}

\character{'Amá} Teteo…
es tiempo.
Es hora de partir.
\action{lo dice con una voz
que sabe cosas que no explica}

\vspace{0.618cm}

\character{Teteo} ¿Cómo?
¿Por qué, 'Amá?
\action{pregunta,
desesperado,
sin entender}

\vspace{0.618cm}

\character{'Amá} Esta es tu última lección, mi'jo.
Al fin,
serás el dios
que siempre vi en ti.

\vspace{0.618cm}

\action{Teteo rompe en llanto.
No entiende lo que ocurre,
pero algo dentro de él
ya lo sabe.
\\
Sabe que es el final
de su madre.}

\vspace{0.618cm}

\character{'Amá} Todo va a estar bien, Teteo.
Viviré en todo.

Estaré en el amor
con el que cuidas a las hormigas.
Estaré cuando bailes
y cuando cantes.
Estaré en las estrellas
que tanto amas.
\action{señala el cielo}

\vspace{0.618cm}

\character{Teteo} ¿Pero por qué tienes que irte?
\action{dice,
limpiándose las lágrimas
con torpeza}

\vspace{0.618cm}

\character{'Amá} Porque incluso los dioses
deben aprender a soltar.

La muerte no es un final.
Es un cruce.
Otro modo de ser.

\vspace{0.618cm}

\character{Teteo} ¿Y yo?
¿Qué hago sin ti?

\vspace{0.618cm}

\character{'Amá} Cuidarás
lo que yo cuidaba.
Serás
el dios
que siempre vi en ti.

\vspace{0.618cm}

\action{'Amá abraza a Teteo
por última vez.
\\
Su cuerpo se vuelve ligero.
Se deshace en el viento,
como si siempre
hubiera sido parte de él.}

\vspace{0.618cm}

\action{Teteo deja correr las lágrimas.
No las detiene.
Las deja ser.
\\
La noche parece no avanzar.
Como si el tiempo
hubiera olvidado moverse.
\\
Y aun así,
muy dentro,
Teteo sabe
que todo estará bien.}


\end{document}
